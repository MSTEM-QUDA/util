%% LyX 1.5.5 created this file.  For more info, see http://www.lyx.org/.
%% Do not edit unless you really know what you are doing.
\documentclass[english]{revtex4}
\usepackage[T1]{fontenc}
\usepackage[latin9]{inputenc}

\makeatletter
%%%%%%%%%%%%%%%%%%%%%%%%%%%%%% User specified LaTeX commands.
 


\makeatother

\usepackage{babel}

\begin{document}

\title{Calculating Ion Populations in an Ideal Plasma}

\maketitle

\section{ The Basics }

Population of an ion ($P_{i}$) in a plasma is the probability of
finding that particle in the state possessing a certain charge, i.
Similarly, if we let $N_{a}$ be the heavy particle density -- the
total number of atoms and ions per unit of volume in a plasma -- then,
$N_{i}=P_{i}*N_{a}$, where $N_{i}$ is the concentration of i-level
ions (ions that lost i electrons), and $P_{i}$, their populations.
Furthermore, since the total of probabilities of all the alternative
events is always equal to one, we get \begin{equation}
\sum_{i=0}^{Z}P_{i}=1,\end{equation}
 where X is the atomic number of the element -- hence the maximum
charge it can attain. Or, similarly: \begin{equation}
\sum_{i=0}^{Z}N_{i}=N_{a},\end{equation}
 which is the same equation multiplied by $N_{a}$. Furthermore, since
in a single-component plasma, each of the $N_{i}$ ions looses i electrons,
which is further true for each of the X types of ions, we obtain the
formula for the total number of free electrons per unit volume, $N_{e}$:
\begin{equation}
N_{e}=\sum_{i=0}^{X}iN_{i},\end{equation}
 which is the condition of plasma neutrality -- the total positive
charge on the ions per unit of volume in a plasma must be equal to
the total number of free electrons in that volume. If we divide this
equation by $N_{a}$, we get: \begin{equation}
<Z>=\sum_{i=0}^{X}iP_{i},\end{equation}
 where <Z> = $N_{e}/N_{a}$ -- the average positive charge per ion
in the plasma (measured in elementary charge units). From all these
formulas we can see that to calculate practically any property of
a plasma, we need to know the populations of all the levels of ions. 


\section{ Boltzmann Distribution}

To calculate the population of an i-level ion at a certain temperature,
we need to know for how many atoms/ions, the energy per atomic cell
is greater than or equal to the energy level of the ith ion. Thus,
we need to know the energy distribution between the atoms and ions
for a given temperature T. Here, recall the Maxwell-Boltzmann Distribution,
which formulates that the probability of finding a particle with certain
energy in some volume of gas (population of that energy state) is
proportional to the Boltzmann factor: \begin{equation}
g_{i}e^{-\frac{E_{i}}{k_{B}T}},\end{equation}
 where the exponent gives the probability of finding a particle in
a specific state with energy $E_{i}$, while $g_{i}$ -- degeneracy
-- is the number of such distinct states that have the same desired
energy. In the case of a plasma, however, the degeneracy value includes
not only the number of states that the i-charged ion might have, but
also the number of distinct states possible for all of the i electrons
that have separated from the original atom. Since each of these electrons
has the same degeneracy value, $g_{e}$, the total number of possible
states for all of these electrons will be $(g_{e})^{i}$. Therefore,
in Eq.(5), $g_{i}$ will be replaced by $g_{i}*(g_{e})^{i}$. Furthermore,
recalling that the sum of the populations of all alternative states
must equal one, we can find the proportionality constant for the Boltzmann
factor. Dividing each of these factors by the value of their sum,
we will normalize them so that their sum after the transformation
will be equal to unity (Eq. 1). Hence, if we also use the ion energy
levels for $E_{i}$, then the ion populations become: \begin{equation}
P_{i}=\frac{g_{i}(g_{e})^{i}e^{-\frac{E_{i}}{k_{B}T}}}{\sum_{i=0}^{X}g_{i}(g_{e})^{i}e^{-\frac{E_{i}}{k_{B}T}}},\end{equation}
 Here, the sum in the denominator is called the partition function.
The main difficulty with finding the actual populations using this
equation comes from the $g_{e}$ and $g_{i}$ values. Through statistical
and quantum mechanics we can find that the value of $g_{e}$ is: \begin{equation}
g_{e}=\frac{2}{N_{e}\Lambda^{3}}\ \ \ \ \ \ \ \ \ \ \ \ \ \Lambda=\sqrt{\frac{h^{2}}{2\pi m_{e}k_{B}T}},\end{equation}
 where $\Lambda$ is the thermal De Broglie wavelength of an electron.
Note that to calculate the value for $g_{e}$, we need to know $N_{e}$,
which is found from Eq.(3) (or 4), which in turn takes the populations
as an input parameter. Hence, we will have to solve a system of equations,
instead of a single equation, to find the correct population values.
This will be addressed later on in the paper. In regards to the $g_{i}$
values, on the other hand, note that in a realistic plasma, while
the value of $g_{e}$ might be of the order of $10^{1}5$, the value
of $g_{i}$ is usually not grater than 3. Hence, after normalization,
the effect of $g_{i}$ on the final population values will be negligibly
small in comparison. Furthermore, since in most cases we are not interested
in the actual population values, but rather in properties of plasma,
such as <Z> (Eq.4), the slight inaccuracies in the populations will
be evened out by the summation often used to calculate such properties,
making the approximation even less evident. Hence, from now on, we
will ignore the $g_{i}$ values in our equations, approximating them
by $g_{i}\approx1$. 


\section{ Saha Ionization Equation and Calculation Speed Improvement}

Another similar approach to finding the population values (if the
value of $g_{e}$ is known) is using the Saha Ionization Equation.
This equation is obtained by taking the ratio of the populations of
the two consecutive level ions as given by Eq.(6): \begin{equation}
\frac{P_{i+1}}{P_{i}}=g_{e}\frac{g_{i+1}}{g_{i}}e^{-\frac{E_{i+1}-E_{i}}{k_{B}T_{e}}}=\frac{2}{N_{e}\Lambda^{3}}\frac{g_{i+1}}{g_{i}}e^{-\frac{I_{i+1}}{k_{B}T_{e}}},\end{equation}
 or, similarly: \begin{equation}
\frac{N_{i+1}N_{e}}{N_{i}}=\frac{2}{\Lambda^{3}}\frac{g_{i+1}}{g_{i}}e^{-\frac{I_{i+1}}{k_{B}T_{e}}},\end{equation}
 which is a better known form of the Saha Equation. Note that here
$I_{i+1}=E_{i+1}-E_{i}$ denotes the ionization energy -- energy required
to remove a single electron from an i-charged ion, forming (i+1)-level
ion. These equations can provide an advantage in finding the populations
if the speed of calculation is important. First, realize that if such
is the case, than it is better to avoid the complex exponents of Eq.(6),
which would take a long time to calculate. To do this, we could instead
start with the natural logarithm of Eq.(8) (now ignoring the ratio
$g_{i+1}/g_{i}\approx1$): \begin{equation}
\ln P_{i+1}-\ln P_{i}=\ln g_{e}-\frac{I_{i+1}}{k_{B}T_{e}},\end{equation}
 Then, we will let $P_{0}=1$ (ln $P_{0}$ = 0), and using this relationship,
build up and array of non-normalized values of $P_{1}$, $P_{2}$,
$P_{3}$, ..., $P_{X}$. The values of this array will vary greatly,
and some will be very large. To avoid having to find exponents of
huge numbers, we can normalize this array so that the population of
the most probable ion $P_{m}ax$ = 1 (ln $P_{m}ax$ = 0). Note that
normalizing an array of logarithms is done by subtraction, rather
than division, since log ($P_{i}$/C) = log $P_{i}$ - log C, and
hence we simply need to subtract ln $P_{m}ax$ from every term. After
this is done, to simplify the calculations even further, we can get
rid of all the values of the array <-5, for example, because the actual
populations corresponding to these values will be negligibly small
-- $P_{i}<e^{-5}=0.0067$, and can be approximated as zeros. Finally,
at this point, it should not take long to find the exponents of the
remaining part of the array, and normalize them as before so that
their sum equals one. This method can then be used instead of Eq.(6)
any time the population values are needed. 


\section{ The System of Equations}

Both of the methods for finding the populations presented above can
only be used if the value of $g_{e}$ is known. However, $g_{e}$
depends on $N_{e}$, which in turn depends on the populations themselves,
and hence we find ourselves faced with solving a system of two equations
-- Eq.(3) (or 4) and Eq.(6) must both be satisfied in order to obtain
the correct population values. To solve this system, we first need
to rewrite the term $g_{e}$ as a function of Z: \begin{equation}
g_{e}=\frac{2}{N_{e}\Lambda^{3}}=\frac{2}{ZN_{a}\Lambda^{3}}=\frac{C_{1}}{Z},\end{equation}
 $C_{1}$, which is independent of Z, is introduced here only to simplify
this ratio. Hence, the Boltmznn factor from Eq.(5) becomes: \begin{equation}
P_{i}\propto\left(\frac{C_{1}}{Z}\right)^{i}e^{-\frac{E_{i}}{k_{B}T_{e}}}=p_{i},\end{equation}
 Since this term appears frequently in the succeeding discussion,
we will call it $p_{i}$ to simplify the notation. Now, we can combine
equations (4) and (6) into a single equation in Z: \begin{equation}
Z=\sum_{i=0}^{X}iP_{i}=\sum_{i=0}^{X}\left(i\frac{p_{i}}{\sum_{j=0}^{X}p_{j}}\right)=\frac{\sum_{i=0}^{X}ip_{i}}{\sum_{i=0}^{X}p_{i}},\end{equation}



\section{ Iterative Approach}

Finally, all the terms in the above equation are known. However, this
expression is too complex to solve by simple algebraic methods and
requires an iterative approach. Direct iterations will converge to
the answer very slowly, if at all. Hence, we will use the Newton-Rapson
iteration method: \begin{equation}
x_{n+1}=x_{n}-\frac{f(x_{n})}{f'(x_{n})},\end{equation}
 This method is used to approximate zeros of the function f. If $x_{n}$
is an approximation of the solution (or a trial solution), than $x_{n+1}$
will be a much better approximation. This method does have limitation
as to whether it will converge to the answer, but if $f'(x_{n})\geq1$,
then convergence is guaranteed. Further, since this method works only
for finding zeros of the function, considering equation (13), let
\begin{equation}
f(Z)=Z-\frac{\sum i\left(\frac{C_{1}}{Z}\right)^{i}e^{-\frac{E_{i}}{k_{B}T_{e}}}}{\sum\left(\frac{C_{1}}{Z}\right)^{i}e^{-\frac{E_{i}}{k_{B}T_{e}}}},\end{equation}
 then when f(Z)=0, Z will have the desired value. Now, to use the
method as in Eq.(14), we need to know f'(Z). Let's first find the
derivative \begin{equation}
\frac{\partial}{\partial Z}\left[\sum\left(\frac{C_{1}}{Z}\right)^{i}e^{-\frac{E_{i}}{k_{B}T_{e}}}\right]=\sum i\left(\frac{C_{1}}{Z}\right)^{i-1}\ \frac{-C_{1}}{Z^{2}}e^{-\frac{E_{i}}{k_{B}T_{e}}}==-\frac{C_{1}}{Z^{2}}\frac{Z}{C_{1}}\sum i\left(\frac{C_{1}}{Z}\right)^{i}e^{-\frac{E_{i}}{k_{B}T_{e}}}=-\frac{1}{Z}\sum ip_{i},\end{equation}
 and similarly: \begin{equation}
\frac{\partial}{\partial Z}\left[\sum ip_{i}\right]=-\frac{1}{Z}\sum i^{2}p_{i},\end{equation}
 Now, combining these answers to find f'(Z), we get: \begin{equation}
\frac{\partial f}{\partial Z}=1-\frac{-\frac{1}{Z}\sum i^{2}p_{i}}{\sum p_{i}}-\frac{-\sum ip_{i}}{\left(\sum p_{i}\right)^{2}}\left(-\frac{1}{Z}\sum ip_{i}\right)==1+\frac{1}{Z}\left(\frac{\sum i^{2}p_{i}}{\sum p_{i}}-\left(\frac{\sum ip_{i}}{\sum p_{i}}\right)^{2}\right),\end{equation}
 Therefore, looking back on Eq.(14) and combining Eq.(15) and (18),
we get the overall resulting equation that can be used to find Z:
\begin{equation}
Z_{n+1}=Z_{n}-\frac{Z_{n}-\frac{\sum ip_{i}}{\sum p_{i}}}{1+\frac{1}{Z}_{n}\left(\frac{\sum i^{2}p_{i}}{\sum p_{i}}-\left(\frac{\sum ip_{i}}{\sum p_{i}}\right)^{2}\right)},\end{equation}
 where the values for $p_{i}$ can be found from equation (12), using
$Z_{n}$ for Z and all sums are across the entire span of possible
ion charges. Further, notice that this equation is equivalent to:
\begin{equation}
Z_{n+1}=Z_{n}-\frac{Z_{n}-<Z>_{n}}{1+\frac{1}{Z}_{n}(<Z^{2}>_{n}-<Z>_{n}\ ^{2})},\end{equation}
 where <Z> is the approximate average charge per ion, and <$Z_{2}$>
is the approximate average of the squares of charges. This equation
can also be used to find Z by first, finding the population values
through either of the presented methods, using $Z_{n}$ for Z in the
calculation of $g_{e}$, and then find the values of $<Z>_{n}$ and
$<Z^{2}>_{n}$ using the formulas: \begin{equation}
<Z>=\sum_{i=1}^{X}iP_{i}\ \ \ \ \ \ \ \ <Z^{2}>=\sum_{i=1}^{X}i^{2}P_{i},\end{equation}
 Also, from Eq.(20), it can be seen that since the average of the
squares is always greater than the square of the average, $<Z^{2}>-<Z>^{2}\geq0$,
and hence $f'(Z)\geq1$, which then necessarily means that the iterations
will converge. Hence, with these equations, we can fairly quickly
find Z, if our initial guess is reasonably close. 


\section{ Initial Approximation}

In order to make the above iterative approach quick and efficient,
it is important to find a decent first approximation for what the
value of <Z> is. Since in most cases in plasma, there is a single
ion whose population is much higher than that of all the others, a
good first guess would be that <Z> is approximately the same as the
charge of that ion. Hence, we need to find the value of i that maximizes
$P_{i}$, or ln $P_{i}$ (which will occur at the same i, but is easier
to find). Recall that at a local extrema of any function, its first
derivative is equal to 0. Since the populations are not defined by
a continuous function, we can similarly say that $\Delta(\ln P_{i})/\Delta i$
is closest to 0 at the maximum value of ln $P_{i}$. Eq.(10) essentially
gives an expression for $\Delta(\ln P_{i})/\Delta i$. Hence, since
we decided to initially approximate $Z\approx i_{P}{}_{max}$, we
can build a sequence of values of what $\Delta(\ln P_{i})/\Delta i$
would be if the current i was equal to $i_{P_{max}}$, and hence to
Z. Hence, we will approximate $g_{e}=C_{1}/Z\approx C_{1}/i$. Further,
when this derivative of the ln $P_{i}$ sequence is $\approx0$, $P_{i}$
will be $\approx P_{i+1}$, and hence i+1 can be used for Z here just
as well. Hence we get: \begin{equation}
\frac{\Delta(\ln P_{i})}{\Delta i}=\frac{\ln P_{i}-\ln P_{i-1}}{1}=\ln(\frac{C_{1}}{i})-\frac{I_{i}}{k_{B}T_{e}},\end{equation}
 Then, the value of i for which the right side of this equation is
closest to 0, will be a reasonable initial approximation for Z. Further,
since, as mentioned, i as well as i-1 could be used for Z here, a
better initial approximation would actually be between the two: $Z\approx i-0.5$.
A separate case arises when the plasma is weak and the vast majority
of the atoms are not ionized ($Z\approx0$). This implies that the
populations of higher level ions will be negligible compared to those
of $P_{0}$, or even $P_{1}$, and hence, Eq.(13) becomes: \begin{equation}
Z=\frac{{\displaystyle \sum_{i=0}^{X}ip_{i}}}{{\displaystyle \sum_{i=0}^{X}p_{i}}}\approx\frac{p_{1}}{p_{0}}=\left(\frac{C_{1}}{Z}\right)e^{-\frac{I_{1}}{k_{B}T_{e}}},\end{equation}
 This can then be easily solved for Z: \begin{equation}
Z\approx\sqrt{C_{1}e^{-\frac{I_{1}}{k_{B}T_{e}}}},\end{equation}
 which is a very good and simple approximation. Hence, when the method
in Eq.(22) yields $i_{P}{}_{max}=1$, the method in Eq.(24) can be
used instead, but only as long as it yields a value for Z less than
1 (if it does not, than $Z\approx1$ and is too large for this method). 


\section{Summary}

Finally, we have concluded that in order to find the ion populations
in a plasma, we can use the following procedure:

\begin{enumerate}
\item Approximate the value of <Z> (average charge per ion) by first, finding
the value of i for which \begin{equation}
\ln(\frac{C_{1}}{i})-\frac{I_{i}}{k_{B}T_{e}}\end{equation}
 is closest to 0, and then approximating $<Z>\approx i-0.5$. Note
that if the i value found is =1, than <Z> might be very close to zero,
in which case use: \begin{equation}
Z\approx\sqrt{C_{1}e^{-\frac{I_{1}}{k_{B}T_{e}}}}\end{equation}


\begin{itemize}
\item $I_{i}$ is the ionization energy (energy required to remove one electron
from (i-1)-level ion, creating i charged ion) 
\item $k_{B}$ is Boltzmann constant;
\item $T_{e}$ is the electron temperature;
\item $C_{1}=2/(N_{a}\Lambda^{3})$ 

\begin{itemize}
\item $N_{a}$ is the total concentration of atoms and ions in the plasma 
\item $\Lambda=\sqrt{\frac{h^{2}}{2\pi m_{e}k_{B}T}}$, which is the thermal
De Broglie wavelength of an electron. 

\begin{itemize}
\item h is Plank's constant 
\item $m_{e}$ is the electron mass 2. 
\end{itemize}
\end{itemize}
\end{itemize}
\item Then, use \begin{equation}
Z_{n+1}=Z_{n}-\frac{Z_{n}-<Z>_{n}}{1+\frac{1}{Z}_{n}(<Z^{2}>_{n}-<Z>_{n}\ ^{2})}\end{equation}
 to iteratively get a better approximation of Z.

\begin{itemize}
\item To find $<Z>_{n}$ and $<Z^{2}>_{n}$, first find the population values
using the equation in the next step, and using $Z_{n}$ for Z. Then,
use the formulas:\begin{equation}
<Z>=\sum_{i=1}^{X}iP_{i}\ \ \ \ \ \ \ \ <Z^{2}>=\sum_{i=1}^{X}i^{2}P_{i}\end{equation}

\end{itemize}
\item Finally, calculate the populations using: \begin{equation}
P_{i}=\frac{\left(\frac{C_{1}}{Z}\right)^{i}e^{-\frac{E_{i}}{k_{B}T_{e}}}}{{\displaystyle \sum_{i=0}^{X}\left(\frac{C_{1}}{Z}\right)^{i}e^{-\frac{E_{i}}{k_{B}T_{e}}}}}\end{equation}

\item If speed of calculation is important, use the method in section 3
(Eq. 10) instead on the previous step.
\end{enumerate}

The table of $Z$ for xenon, in wide ranges of temperature and atomic 
concentrations are given in the following table
\input{Table1}
Note that the unphysical negatize values of $Z$ markes the parameters domain, 
in whith the electron degeneracy is not negligible 

\end{document}
