\documentclass[12pt]{article}

\newcommand{\Eint}{E_{\rm I}}
\newcommand{\Erad}{E_{\rm R}}
\newcommand{\prad}{p_{\rm R}}
\newcommand{\kappaP}{\kappa_{\rm P}}


\begin{document}
\section{Solution Method for the Radiation Hydrodynamcs with Grey-Diffusion approximation}  
The governing equations of radiation hydrodynamics with grey-diffusion 
approximation can be obtained in the first approximation in u/c. They 
express the near conservation of mass, momentum, total energy, and 
radiation energy: 
\begin{equation}\label{eq:density} \frac{\partial\rho}{\partial t} +
\nabla\cdot\left[ \rho {\bf u} \right] = 0, 
\end{equation} 
\begin{equation}   \frac{\partial}{\partial t}\left( \rho {\bf u} \right) +
 \nabla\cdot\left[ \rho{\bf u}{\bf u}     +(p+\prad){\bf I} \right] = 0, 
\end{equation} 
\begin{equation}   \frac{\partial ({\cal E} + \Erad)}{\partial t} + 
\nabla\cdot   \left[ \left( {\cal E} + \Erad+ p+\prad \right) {\bf u} 
\right]   = \nabla\cdot\left[ \frac{c}{ 3 \overline{\chi} } \nabla \Erad 
\right], 
\end{equation}\label{eq:energy} 
\begin{equation}\label{eq:grey}   \frac{\partial \Erad}{\partial t} + 
\nabla\cdot \left[ \Erad {\bf u} \right]   + \prad\nabla\cdot{\bf u} 
=    \nabla\cdot\left[ \frac{c}{ 3\overline{\chi} } \nabla \Erad \right]   
-\kappaP c (\Erad - \frac{4\sigma}{c} T^4), 
\end{equation} 
where $\rho$, ${\bf u}$, $p$, $T$ are the density, plasma velocity, gas 
kinetic pressure, and temperature, respectively. The total plasma energy 
density ${\cal E}$ is related to the internal plasma energy $e$: 
\begin{equation} 
{\cal E} = \frac{1}{2}\rho u^2 + \rho e. 
\end{equation} 
The radiation field is assumed to be isotropic, so that the radiation 
pressure can be obtained from the radiation energy density $\Erad$: 
\begin{equation} 
\prad = \frac{1}{3}\Erad. 
\end{equation} 
The radiation is treated as a fluid that carries momentum and energy. 
In essence, equations (\ref{eq:density})--(\ref{eq:grey}) describe a two 
temperature fluid. Two cross-sections have been introduced, namely the 
Planck mean opacity $\kappa_{\rm P}$ and an averaged opacity 
$\overline{\chi}$ that appears in the radiation diffusion coeffient.  
To close the dynamical equations we need equation of state data. 
If the radiation is negligible and the material is a polytropic gas, 
then the internal energy would be 
\begin{equation} 
\rho e = \frac{p}{\gamma - 1}, 
\end{equation} 
where $\gamma$ is fixed. In general, we can not fix the polytropic index due 
to radiation effects. In the following we will fix $\gamma$, but instead  
indicate the deviation in the internal energy from that of a polytropic 
hydro-gas by $\rho e$: 
\begin{equation} 
\rho e = \frac{p}{\gamma - 1} + \Delta (\rho e). 
\end{equation} 
This introduces however a new advection equation for this extra internal 
energy of the plasma 
\begin{equation} 
\frac{\partial}{\partial t} \Delta(\rho e) + \nabla\cdot\left[  
\Delta(\rho e) {\bf u} \right] = 0, 
\end{equation} 
which we need to solve together with the other radiation hydrodynamics 
equations.  The problem will be solved using the shock-capturing schemes of 
the BATSRUS code. The left hand side of equations 
(\ref{eq:density})--(\ref{eq:grey}) are in a form that resembles the pure 
hydro equations. We will fully exploit this feature, so that we can solve 
this with the hydro solvers of BATSRUS. For the analogy with the hydro 
equations with have to re-interpret in the hydro solver the pressure with 
the total gas kinetic and radiation pressure $p+\prad$. The internal 
energy is now given by the total gas and radiation energy 
\begin{eqnarray} \rho e &=& \frac{p}{\gamma - 1} + \Delta (\rho e) + 
\Erad, \nonumber \\        
&=& \frac{p+\prad}{\gamma - 1} + \Delta (\rho e) + \frac{3}{2}\prad, 
\end{eqnarray} 
where we have fixed $\gamma=5/3$. This indicates that we have to solve yet 
another advection equation for $3\prad/2=\Erad/2$. This is however 
already accomplished by the left hand side of the radiation energy equation 
(\ref{eq:grey}).  The solution of our radiation hydrodynamics equations with 
grey diffusion approximation is obtained as follows: 
\begin{itemize} 
\item First solve the hyperbolic part of the equations, which now read 
\begin{equation}\label{eq:hydro1} \frac{\partial\rho}{\partial t} +
\nabla\cdot\left[ \rho {\bf u} \right] = 0, \end{equation} 
\begin{equation}\label{eq:hydro2}   \frac{\partial}{\partial t}\left( 
\rho {\bf u} \right) + \nabla\cdot\left[ \rho{\bf u}{\bf u}     
+(p+\prad){\bf I} \right] = 0, 
\end{equation} 
\begin{equation}\label{eq:deficit1}   
\frac{\partial}{\partial t} \Delta(\rho e) + \nabla\cdot\left[  
\Delta(\rho e) {\bf u} \right] = 0, 
\end{equation} 
\begin{equation}\label{eq:deficit2}   
\frac{\partial \Erad}{\partial t} + \nabla\cdot \left[ \Erad {\bf u} 
\right]   + \prad\nabla\cdot{\bf u} = 0, 
\end{equation} 
\begin{equation}\label{eq:hydro3}   \frac{\partial}{\partial t} 
(\frac{1}{2}\rho u^2 + \frac{p+\prad}{\gamma - 1}) + \nabla\cdot   
\left[ \left( \frac{1}{2}\rho u^2 + \frac{p+\prad}{\gamma - 1} + 
p+\prad \right) {\bf u} \right]   - 
\frac{1}{2}\prad\nabla\cdot{\bf u} = 0, 
\end{equation} 
where equations (\ref{eq:hydro1}), (\ref{eq:hydro2}), and (\ref{eq:hydro3}) 
are solved by the hydrodynamic numerical scheme and the energy equations 
(\ref{eq:deficit1}) and (\ref{eq:deficit2}) are treated as advected scalar 
equations. The overall system contains two sound waves that are modified by 
the radiation pressure. Since we have chosen to fix $\gamma$ on the maximum 
allowable value of 5/3, we can easily find an upper bound for the sound speed: 
\begin{equation}  
c^2=\frac{\gamma (p+\prad)}{\rho}. 
\end{equation} 
Using this wave speed for the numerical diffusion and determining the time 
step of the hyperbolic part of the equations helps to stabilize the scheme.  
\item The previous step provides an intermediate solution for the radiative 
energy density, denoted as $\Erad^\prime$. Using the deficit in the 
internal energy for the plasma and radiation energy density as found by 
equation (\ref{eq:deficit1}) and (\ref{eq:deficit2}), we can recover the 
true internal energy of the plasma (denoted as $e^\prime$). By applying the 
equation of state for our materials of choice we obtain the updated plasma 
pressure and temperature.   
\item In the next stage we have to solve for the source terms in the energy 
equations. This amounts to solving a coupled system for the plasmas 
temperature and radiation temperature ($\Erad \propto T_{\rm R}^4$): 
\begin{eqnarray}
\rho \frac{\partial e(T)}{\partial t} = 
 \kappaP c (\Erad - \frac{4\sigma}{c} T^4), \label{eq:eint}\\ 
\frac{\partial \Erad}{\partial t} = \nabla\cdot\left[ 
  \frac{c}{ 3\overline{\chi} } \nabla \Erad \right]   - 
 \kappaP c (\Erad - \frac{4\sigma}{c} T^4)  \label{eq:erad}
\end{eqnarray} 
and advance solution through time step with initial conditions 
$e^\prime$ for the plasma internal energy and $\Erad^\prime$ for the 
radiation energy.  
\end{itemize}  

\section{Semi-Implicit Scheme}

We describe here the scheme to solve equations 
(\ref{eq:eint}) and (\ref{eq:erad}) implicitly. 
Discretizing in time leads to
\begin{eqnarray}
\Eint^{n+1} &=& \Eint^* + \Delta t K^*\left[\Erad^{n+1} - (aT^4)^{n+1}\right]  
   \label{eq:Eint2}\\
\Erad^{n+1} &=& \Erad^* - \Delta t K^*\left[\Erad^{n+1} - (aT^4)^{n+1}\right]
                + \Delta t \nabla\cdot\left[ D^* \nabla \Erad^{n+1} \right]
   \label{eq:Erad2}
\end{eqnarray}
where time level $*$ corresponds to the state after the hydro update,
and we introduced the following notation for
internal energy, energy coupling, radiation diffusion and radiation
energy coefficients:
$\Eint = \rho e$, $K = \kappaP c$, $D = \frac{c}{ 3\overline{\chi}}$,
and $a = 4\sigma c$.
Notice that the coupling and diffusion coefficients are taken at time
level $*$ (frozen coefficients). This leads to a temporally first order
scheme in general (unless the coefficients are constants in time).
One can either 
\begin{enumerate}
\item solve the coupled system of equations (\ref{eq:Eint2})
      and (\ref{eq:Erad2}) implicitly or 
\item solve equation (\ref{eq:Eint2}) for $\Eint^{n+1}$,
      substitute the solution back into (\ref{eq:Erad2}), 
      and solve the resulting scalar equation (\ref{eq:Erad2})
      implicitly.
\end{enumerate}
Both approaches involve the linearization of the $\Eint(T)$ function.

Here we describe the second scheme, because it is more efficient. 
Note that if we had heat conduction in (\ref{eq:Eint2}), 
then we would have to solve the coupled system of equations.

First we introduce the Planck function $B = a T^4$ as a new variable,
and replace $\Eint$ with it using the chain rule
\begin{equation}
\frac{\partial \Eint}{\partial t} = 
   \frac{\partial \Eint}{\partial T}
   \frac{\partial T}{\partial B}
   \frac{\partial B}{\partial t}
   = \frac{c_V}{4 a T^3}\frac{\partial B}{\partial t}
\end{equation}
Now equation (\ref{eq:Eint2}) can be replaced with
\begin{equation}
B^{n+1} = B^* + \Delta t K'\left[\Erad^{n+1} - B^{n+1}\right]  
   \label{eq:B}
\end{equation}
where
\begin{equation}
K' = K^* \frac{4 a T^3}{c_V}
\end{equation}
Equation (\ref{eq:B}) can be solved for
\begin{equation}
  B^{n+1} = \frac{B^* + \Delta t K' \Erad^{n+1}}{1 + \Delta t K'}
   \label{eq:B2}
\end{equation}
This can be substituted into equation (\ref{eq:Erad2}) to obtain
\begin{equation}
\Erad^{n+1} = \Erad^* - \Delta t K^*\left[\Erad^{n+1} 
              - \frac{B^* + \Delta t K' \Erad^{n+1}}{1 + \Delta t K'}\right]
                + \Delta t \nabla\cdot\left[ D^* \nabla \Erad^{n+1} \right]
\end{equation}
and simplified to
\begin{equation}
\Erad^{n+1} = \Erad^* - \Delta t K''\left[\Erad^{n+1} - B^*\right]
                + \Delta t \nabla\cdot\left[ D^* \nabla \Erad^{n+1} \right]
\label{eq:Erad3}
\end{equation}
where
\begin{equation}
K'' = \frac{K^*}{1 + \Delta t K'} 
\end{equation}
It is convenient to introduce $\Delta \Erad = \Erad^{n+1} - \Erad^*$ and
rearrange equation (\ref{eq:Erad3}) as
\begin{equation}
\left[\frac{1}{\Delta t} + K'' - \nabla\cdot D^* \nabla\right] \Delta \Erad 
   = - K''\left[\Erad^* - B^*\right]
                + \nabla\cdot\left[ D^* \nabla \Erad^* \right]
\label{eq:DeltaErad}
\end{equation}
We solve equation (\ref{eq:DeltaErad}) for $\Delta \Erad$
using a linear solver, update $\Erad^{n+1} = \Erad^* + \Delta \Erad$
and then either
\begin{enumerate}
\item obtain $B^{n+1}$ from (\ref{eq:B2}), and then get
 $T^{n+1}$ and $\Eint^{n+1}$, or
\item use a conservative update for the internal energy
\end{enumerate}
The second choice is
\begin{equation}
\Eint^{n+1} = \Eint^n + \Delta t K''\left[\Erad^{n+1} - B^*\right]
\end{equation}
which conserves the total energy to round-off error.

\newpage

\subsection{Crank-Nicolson scheme}

We can attempt to go second order in time with the assumption of
(temporally) constant coefficients using the Crank-Nicolson scheme
with $\beta=1/2$:
\begin{eqnarray}
B^{n+1} &=& B^* 
   + \beta \Delta t K'\left[\Erad^{n+1} - B^{n+1}\right]  
   + (1-\beta)\Delta t K'\left[\Erad^* - B^*\right]  
   \label{eq:B4}\\
\Erad^{n+1} &=& \Erad^* 
   - \beta \Delta t K^*\left[\Erad^{n+1} - B^{n+1} \right]
   - (1-\beta) \Delta t K^*\left[\Erad^* - B^*\right] \nonumber\\
   && + \beta\Delta t \nabla\cdot\left[ D^* \nabla \Erad^{n+1} \right]
      + (1-\beta)\Delta t \nabla\cdot\left[ D^* \nabla \Erad^* \right]
   \label{eq:Erad4}
\end{eqnarray}
We can solve (\ref{eq:B4}) for 
\begin{equation}
  B^{n+1} = \frac{B^* + \beta \Delta t K' \Erad^{n+1}
                  + (1-\beta)  \Delta t K' (\Erad^* - B^*)}
                 {1 + \beta\Delta t K'}
   \label{eq:B5}
\end{equation}
This can be substituted into equation (\ref{eq:Erad4}) to obtain
\begin{eqnarray}
\Erad^{n+1} &=& \Erad^* - \beta \Delta t K^*\left[\Erad^{n+1} 
              - \frac{B^* + \beta \Delta t K' \Erad^{n+1}
                      + (1-\beta)  \Delta t K' (\Erad^* - B^*)}
                     {1 + \beta\Delta t K'}\right] \nonumber\\
  && - (1-\beta) \Delta t K^*\left[\Erad^* - B^*\right] \nonumber\\
  && +     \beta \Delta t \nabla\cdot\left[ D^* \nabla \Erad^{n+1} \right]
      + (1-\beta)\Delta t \nabla\cdot\left[ D^* \nabla \Erad^* \right]
\end{eqnarray}
and can be simplified to
\begin{eqnarray}
\Erad^{n+1} &=& \Erad^* - \beta \Delta t K''\left[\Erad^{n+1} 
              - B^*  \right] 
    - (1-\beta) \Delta t K'' \left[\Erad^* - B^*\right] \nonumber\\
  && +     \beta \Delta t \nabla\cdot\left[ D^* \nabla \Erad^{n+1} \right]
      + (1-\beta)\Delta t \nabla\cdot\left[ D^* \nabla \Erad^* \right]
  \label{eq:Erad5}
\end{eqnarray}
where
\begin{equation}
K'' = \frac{K^*}{1 + \beta \Delta t K'}  \label{eq:K5}
\end{equation}
Equation (\ref{eq:Erad5}) can be rearranged to
\begin{equation}
\left[\frac{1}{\Delta t} + \beta K'' - \beta \nabla\cdot D^* \nabla\right] 
   \Delta \Erad 
   = - K''\left[\Erad^* - B^*\right]
                + \nabla\cdot\left[ D^* \nabla \Erad^* \right]
\label{eq:DeltaErad2}
\end{equation}
Note that the only difference relative to the backward Euler scheme is
the presence of $\beta$ in equations (\ref{eq:DeltaErad2}) and (\ref{eq:K5}). 

Equation (\ref{eq:DeltaErad2}) is solved for $\Delta \Erad$,
and we update $\Erad^{n+1} = \Erad^* + \Delta \Erad$, one can 
either calculate $B^{n+1}$ from equation (\ref{eq:B5}), 
or do a conservative update for the internal energy
\begin{equation}
  \Eint^{n+1} = \Eint^n + \beta \Delta t K''\left[\Erad^{n+1} - B^*\right]
                        + (1-\beta) \Delta t K''\left[\Erad^* - B^*\right]
\end{equation}

\end{document}  
