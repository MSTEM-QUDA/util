\section{Excited states of atoms and ions}
{\bf To account for excitation of atoms and ions} we need to involve the
distribution not only over theionization states, $i$, but also to quantify the
ion distribution over the ground and 
excitated levels. 

For simplicity,  multiple excitation and autoionization are neglected. We
count as the separate excited levels only those ones for which the principal 
quantum number, 
$n$, of the outermost electron, exceeds that for the atom or ion in its ground state, 
$n_{gr}$, at least by one. The excited states, therefore, can be enumerated using two indexes,
namely, the ion charge, $i=0,1,2...$, and the principal quantum number 
$n(i)=n_{gr}(i),n_{gr}(i)+1,...$. 

The partition function, $p_i$, describing the ion distribution over the charge state $i$ 
(recall that $\sum_i{p_i}=1$), 
is now split into smaller populations, $p_{i,n}$, with each of them relating to a particular 
excitation level, $n$. Note that:
\begin{equation}
\sum_n{p_{i,n}}=p_i,\qquad \sum{\frac{p_{i,n}}{p_i}}=1.	
\end{equation}
That is why the statistical weights, $w_{i,n}$, and, hence, the statistical sum and the partition functions
become more complex:
\begin{equation}
w_{i,n} = g_{i,n} \cdot g_e^i \exp \left( -\frac{E^*_i + E^{exc}_{i,n}}{T} \right),\qquad
S = \sum_{i=0}^{i_{max}} \sum_{n=n_{gr}}^{\infty} w_{i,n},\qquad
p_{i,n} = \frac{w_{i,n}}{S},
\end{equation}
where $g_{i,n}$ stands for the excitation level degeneracy.

The excitation energy, $E^{exc}_{i,n}$, we calculate within a Bohr-like model, which is expressed in
Eq.(2.9) from \cite{ionmix}:
\begin{equation}
E^{exc}_{i,n} = I_i \cdot n_{gr}^2 \left( \frac1{n_{gr}^2} - \frac1{n^2} \right), \qquad
g_{i,n} = 2 n^2.
\end{equation}

Up to this point mean values and covariances were being calculated over $i_{max}+1$ possible
values of $i$. But now $i$ is not the only variable the energy levels depend on.
Accounting for excitation levels the way described above leads us to energy levels
being a function of $i$ and $n$.
This only increases the number of terms in the statistical sum, but does not change
anything in principle. To obtain the thermodynamic variables and the thermodynamic functions
we can just replace the mean values over $i$ with mean values over $i$ and $n$ in the
Eqs.(\ref{generalE}), (\ref{generalP}), (\ref{generalCv}),
(\ref{generalPT}), (\ref{generalCompr}) and substitute the new expression for energy levels.

The excitation energy, $E^{exc}_{i,n}$, does not contribute to $\partial E / \partial V$,
therefore substitution of the new energy levels, $E_{i,n}$, into
Eqs.(\ref{madE}), (\ref{madP}), (\ref{madCv}), (\ref{madPT}), (\ref{madCompr})
gives the correct formulas for the model of excitation levels.
