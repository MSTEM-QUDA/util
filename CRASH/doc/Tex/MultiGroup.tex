%  Copyright (C) 2002 Regents of the University of Michigan, portions used with permission 
%  For more information, see http://csem.engin.umich.edu/tools/swmf
% LyX 1.5.5 created this file.  For more info, see http://www.lyx.org/.
%% Do not edit unless you really know what you are doing.
%\documentclass[english,12pt]{revtex4}
%\usepackage[T1]{fontenc}
%\usepackage[latin9]{inputenc}
%\usepackage{nicefrac}
%\usepackage{graphicx}
%\usepackage{multirow}
%\usepackage{array}
%\makeatletter
%%%%%%%%%%%%%%%%%%%%%%%%%%%%%% LyX specific LaTeX commands.
%% Because html converters don't know tabularnewline

%%%%%%%%%%%%%%%%%%%%%%%%%%%%%% User specified LaTeX commands.
%% LyX 1.5.5 created this file.  For more info, see http://www.lyx.org/.
%% Do not edit unless you really know what you are doing.
%\usepackage{nicefrac}
\large{\bf PART 2. RADIATION TARNSPORT IN DENSE PLASMAS}
%\begin{document}
\section{Multi-group diffusion: governing equations and general ralationships}
{\bf Governing equation} to describe the radiation transport in the multigrop diffusion 
approximation may be written as the partial differential equation for the spectral 
energy density, $E_{\varepsilon}$, which is related to the unit ov volume and the unit
interval of the photon energy, $\varepsilon$. The energy is assumed to be integrated 
over the solid angle of directions of the photon propagation. Once  the spectral energy
is integrated 
density over photon energies, the total radiation energy density is obtained:
\begin{equation}\label{eq:mg0}
E=\int_0^\infty{E_\varepsilon d\varepsilon}.
\end{equation}

The governing equation for the spectral energy density is as follows:
\begin{equation}\label{eq:mg1}
\frac{\partial E_\varepsilon}{\partial t}+\nabla\cdot({\bf u}E_\varepsilon)-(\gamma_R-1)(\nabla\cdot{\bf u})\varepsilon\frac{\partial E_\varepsilon}{\partial \varepsilon}=
{\rm diffusion + emission - absorption}.
\end{equation}
The second and third terms in the left hand side of Eq.(\ref{eq:mg1}) express the time evolution of the spectral energy density resulting from: (1) the radiation 
advection and comression with the background, which moves with the velocity, ${\bf u}$; as well as (2) the photon systematic blue (red) shift in the convergent (divergent) 
motions, which is analogous to the first order Fermi acceleration of chanrged particles in a moving plasma with the frozen in magnetic field. 
Herwith $\gamma_R=4/3$ is the adiabat index of a relativistic (photon) gas. The processes described by the symbolic terms in the right hand side of Eq.(\ref{eq:mg1})
are described below.

{\bf A set of multi-group equations} may be indroduced when we choose a set of frequency groups. Here we enumerate groups with the index, $g=1, G$. The interval of the 
photon energies, relating to the $g$th group is denoted as $[\varepsilon_{g-1/2},\varepsilon_{g+1/2}]$. The discrete set of unknowns, $E_g$, is introduced in terms of the
integrals of the spectral energy density of the frequency group interval:
\begin{equation}\label{eq:mg2}
E_g=\int_{\varepsilon_{g-1/2}}^{\varepsilon_{g+1/2}}{E_\varepsilon d\varepsilon}. 
\end{equation}
Note that according to Eqs.(\ref{eq:mg0},\ref{eq:mg2})
\begin{equation}
E=\sum_g{E_g}.
\end{equation}

Although some of the formulae below are not sensitive to the choice of the group set, here we specify the boundaries of the frequency groups to be such that 
{\it the frequency logarithm} is equally spaced (rather than the frequency itself):
\begin{equation}\label{eq:mg3}
\log(\varepsilon_{g+1/2}) - \log(\varepsilon_{g-1/2}) = \Delta(\log \varepsilon) = const.
\end{equation} 
Note that as long as the number of groups, $G$, tends to infinity, the ratio $E_g/\Delta(\log \varepsilon)$ tends to the local value of 
$\varepsilon E_\varepsilon$, rather than to $E_\varepsilon$. Therefore, the grequency integrals on the equally spaced logarithmic frequency 
grid allow us to approximate not
a spectral energy density, but its product by the photon energy:
\begin{equation}\label{eq:mg4}
\frac{E_g}{ \Delta(\log \varepsilon)}\approx \varepsilon E_\varepsilon.
\end{equation}

Now we can integrate Eq.(\ref{eq:mg1}) to arrive at the desired set of the multigroup equations:
\begin{eqnarray}
\frac{\partial E_q}
{\partial t}&+&
\nabla\cdot({\bf u}E_g)+
(\gamma_R-1)(\nabla\cdot{\bf u})E_g+\nonumber\\
&+&\left[-(\gamma_R-1)(\nabla\cdot{\bf u})\right]\left[
\varepsilon_{g+1/2} E_\varepsilon
(\varepsilon_{g+1/2})-
\varepsilon_{g-1/2} E_\varepsilon(\varepsilon_{g-1/2})\right]
=\nonumber\\
&=&
\int_{\varepsilon_{g-1/2}}^{\varepsilon_{g+1/2}}
{
{\rm (diffusion + emission - absorption)}
d\varepsilon}.\label{eq:mg5}
\end{eqnarray}
If the number of frequency groups, $G$ is sufficiently large, we employ the approximation as in Eq.(\ref{eq:mg4}), which allows us to close Eq.(\ref{eq:mg5}) in the 
following form:
\begin{eqnarray}
\frac{\partial E_q}
{\partial t}&+&
\nabla\cdot({\bf u}E_g)+
(\gamma_R-1)(\nabla\cdot{\bf u})E_g+
%\nonumber\\&+&
\frac{-(\gamma_R-1)(\nabla\cdot{\bf u})}{ \Delta(\log \varepsilon)}\left[
E_{g+1/2}- E_{g-1/2})\right]
=\nonumber\\
&=&
\int_{\varepsilon_{g-1/2}}^{\varepsilon_{g+1/2}}
{
{\rm (diffusion + emission - absorption)}
d\varepsilon}.\label{eq:mg6}
\end{eqnarray}
where the values $E_{g\pm1/2}$ should be interpolated from the mesh-centered values $E_g$ towards the frequency values corresponding to the inter-group boundary.

Note that we arrived to difference-differential equation, with the left-hand side including: (1) the conservative advection of the radiation energy density with
the velocity ${\bf u}$; the work done by the radiation pressure $P_g=(\gamma_R-1)E_g$; (3) and, as a new element, a linear conservative advection with respect to
the log-frequency coordinate. The flux-to-control-volume ratio for the latter effect equals 
$$F_{g-1/2}=-(\gamma_R-1)(\nabla\cdot{\bf u})E_{g-1/2}/ \Delta(\log \varepsilon).$$ 
Eq.(\ref{eq:mg6}) provides us a ready numerical scheme to solve this advection numerically. The recipi to construct a numerical flux for a linear advection equation
are well-known. Specifically, we apply a limited reconstruction procedure, with the 
'superbee' limiter function,
$$L(a,b)=\frac12[{\rm sign}(a)+{\rm sign}(b)]\min(\max(|a|,|b|),2|a|,2|b|),$$ to obtain left and right interpolated values, 
$$E^{(L)}_{g-1/2}=E_{g-1}+\frac12L(E_g-E_{g-1},E_{g-1}-E_{g-2}),$$
$$ E^{(R)}_{g-1/2}=E_g-\frac12L(E_{g+1}-E_{g},E_{g}-E_{g-1})$$. 
With these interpolated values  the upwinded numerical flux is constructed as 
follows:
\begin{eqnarray}
F_{g-1/2}&=&-(\gamma_R-1)(\nabla\cdot{\bf u})E^{(R)}_{g-1/2}/ \Delta(\log \varepsilon),\qquad (\nabla\cdot{\bf u})\ge0,\nonumber\\
F_{g-1/2}&=&-(\gamma_R-1)(\nabla\cdot{\bf u})E^{(L)}_{g-1/2}/ \Delta(\log \varepsilon),\qquad (\nabla\cdot{\bf u})\le0.\label{eq:mg7}
\end{eqnarray}
Eqs.(\ref{eq:mg6}) with the numerical fluxes as in Eq.(\ref{eq:mg7}) may be further 
discretized using the control volume methods and solved numerically, assuming the 
right hand side to be zero, as the coupled element of the system of hydrodynamic 
(or magneto-hydro-dynamic) equations involving the ratiation pressure contributions. 
Note, that suming up Eqs.(\ref{eq:mg6}) we obtain the equation of the gray radiation
diffusion:
\begin{equation}\label{eq:mg8}
\frac{\partial E}{\partial t}+\nabla\cdot({\bf u}E)+(\gamma_R-1)(\nabla\cdot{\bf u}) E=\int_0^\infty{
{(\rm diffusion + emission - absorption)}d\varepsilon}.
\end{equation}
If, again, we assume the right hand side of Eq.(\ref{eq:mg8}) to be zero, then only Eq.(\ref{eq:mg8}) is two-way coupled to the system of equation of the plasma motion,
since in these equations the radiation effects the plasma motion only via the total radiation pressure:
\begin{equation}
P=(\gamma_R-1)E.
\end{equation}
On advancing the solution of this coupled system through the time step, the numerical solution of each of Eqs.(\ref{eq:mg6}) may be also advanced through the time step, 
$\Delta t$, as 
long as ${\bf u}$ is known. Depending on the value of $\Delta(\log \varepsilon)$ one may prefer to treat the advection over frequency as: (1) an extra flux (the control
volume in this case is treated as the four-dimentional rectangular box $\Delta x*\Delta y*\Delta z*\Delta (\log \varepsilon)$); or (2) an extra advance operator, which can be
split out and hanled separately. In the second case the second order of accuracy, $o((\Delta t)^2)$, may be achieved with the choice of the numerical flux as follows: 
\begin{eqnarray}
\Delta t F_{g-1/2}&=&-{\rm CFL}\left[E_g-\frac{1-{\rm CFL}}2L(E_{g+1}-E_{g},E_{g}-E_{g-1})\right],\qquad (\nabla\cdot{\bf u})\ge0,\nonumber\\
\Delta t F_{g-1/2}&=&{\rm CFL}\left[E_{g-1}+\frac{1-{\rm CFL}}2L(E_g-E_{g-1},E_{g-1}-E_{g-2})\right],\qquad (\nabla\cdot{\bf u})\le0.\label{eq:mg9}
\end{eqnarray}
where
\begin{equation}
{\rm CFL}=\frac{(\gamma_R-1)\Delta t}{\Delta(\log \varepsilon)}|\nabla\cdot{\bf u}|\ge 0
\end{equation}
is the Courant-Friedrichs-Levi number.
\section{Absroption, emission and stimulated emission}
An account of the stimuated emission is not less important in the context of the multi-group radiation diffusion than the use of the locat 
termodynamic equailibrium assumption. As long as the stimulated emission is not often discussed in books regarding the diffisive radiation 
transport, the code developer may meet a problem while bridging from the absorption coefficient, $a_\varepsilon$, 
which is calculated from single-photon diagrams of Quantum ElectroDynamics (QED), describing an absorption of a single photon, to the absorption 
coefficient which is to be used in the radiation trasport simulations. Below we follow \cite{ZR}, in which this subject is presented quite 
transparently.

Consider a part of the radiation spectrum, of a small width of $\Delta\varepsilon$, about the photon energy, $\varepsilon$, which is resonant with some 
bound-bound transition, that is 
\begin{equation}\label{eq:mg10}
\varepsilon={\cal E}_E - {\cal E}_A,
\end{equation}
where ${\cal E}_A$ and ${\cal E}_E$ are some energy states of an atom (or ion, below we refer the system in these states to as an Emitter or an Absorber). For 
the sake of simplicity, assume for a while that the degeneracy
(multiplicity) is equal to one for both upper and lower level:
$$
g_E = 1,\qquad g_A = 1,
$$  
the use of the same denotation for the degeneracy as for the group index should not confuse the reader.

Assume there is no photon in the initial state. The only radiation process which can occur in this case is the spontaneous emission. Introduce 
the probability, $dw_p/d{\bf o}$, for the spontaneous emission from a single emitter, into the element of a solid andgle, with a given polarization 
of the photon, $p$, per a unit of time.  The total spontaneous emission from the unit of volume is, hence, 
$$N_E\frac{dw_p}{d{\bf o}}d{\bf o},$$ 
where $N_{E,A}$ are the abundancies of the emitters and the absorbers correspondingly. The contribution from the 
spontaneous emission to Eq.(\ref{eq:mg2}) should have an extra factor of $4\pi$ (due to the integration over the photon directions) 
times $2$ (summation by polarizations) times $\varepsilon$ (an emitted energy per a photon) divided by $\Delta\varepsilon$.

Now assume that there are $N_p$ photons in the considered unit volume of plasma, which are in the same as the emitted photon. The rules to express the 
probability of stimulated emission and absorption may be found in any textbook on QED,
particularly, we follow \cite{lp}. The probability of a total emission (spontaneous plus stimilated) equals 
$$(1+N_p)\frac{dw_p}{d{\bf o}}$$ 
while the number of the absorbed photons from the same photon state, per an absorber, equals
$$
(1+N_p)\frac{dw_p}{d{\bf o}}
$$
On integrating the total absorbtion and emission by $d{\bf o}$ and suming the result up by polarizations, we arrive at the following expression for
the emission and absoption terms in Eq.(\ref{eq:mg2}):
$$
{\rm emission - absortion}=\frac{8\pi\varepsilon}{\Delta\varepsilon}\frac{dw_p}{d{\bf o}}(N_E(1+N_p)-N_AN_p)
$$
Now we may abandon the assumption about non-degeneracy of the emeitter and absorber states. The extra freedom of the ssytem in its {\it final} state to occupy any 
of $g_f$ final states simply multiplies the transition probability by a factor of $g_f$. Properly applying the factors $g_f=g_E$ and $g_f=g_A$, we find:
$$
{\rm emission - absortion}=\frac{8\pi\varepsilon}{\Delta\varepsilon}\frac{dw_p}{d{\bf o}}(N_Eg_A(1+N_p)-N_Ag_EN_p).
$$

The population of the photon state, $N_p$, may be related to the spectral energy density. A radiation energy in the unit volume within the interval of 
the photon energies, $d{\varepsilon}$, may be written as $E_\varepsilon d\varepsilon$. The same energy may be obtained if we multiply $2N_p\varepsilon$ 
(the factor of two accounts for two polarizations of the photon) by the number of photon states within $d\varepsilon$. In the unit volume the number of photon states
per the pase volume of $dk_xdk_ydk_z$ equals $dk_xdk_ydk_z/(2\pi)^3$, with $k_x$, $k_y$, $k_z$ being the three components of the wave vector. The number of photon states 
per the interval of $dk$ is, hence, $4\pi k^2dk/(2\pi)^3=4\pi \varepsilon^2d\varepsilon/(hc)^3$. We find:
$$
E_\varepsilon=\frac{8\pi N_p\varepsilon^3}{h^3c^3},
$$
and
\begin{equation}
{\rm emission - absorption}=
ca_\varepsilon\left[\frac{N_Eg_A}{N_Ag_E}\left(\frac{8\pi\varepsilon^3}{h^3c^3}+E_\varepsilon\right) -E_\varepsilon\right].
\end{equation}
where the absorption coefficient is introduced (the radiation energy dissipation per a unit of length), which is related to the emission probability as follows:
\begin{equation}
a_\varepsilon=\frac{h^3c^2N_Ag_E}{8\pi \varepsilon^2}\frac{dw_p}{d{\bf o}\Delta\varepsilon}.
\end{equation}

Under the condition of a local thermodynamic equilibrium, the partition function of atoms and ions in different charge and energy states is governed by the Boltzmann 
statistics, so that the abundance of each excited level is proportianal to its mutiplicity and to the Boltzbann factor, $\exp(-E/(k_BT))$. Therefore, under these 
circumstances, using Eq.(\ref{eq:mg10}), we find
\begin{equation}\label{eq:mg11}
N_{E,A}\propto g_{E,A}\exp[-E_{E,A}/(k_BT)],\qquad \frac{N_Eg_A}{N_Ag_E}=\exp\left[\frac{E_A-E_E}{k_BT}\right]=\exp[-\varepsilon/(k_BT)]
\end{equation}  
and
\begin{equation}
{\rm emission - absorption}=
ca^\prime_\varepsilon\left(\frac{8\pi}{h^3c^3}\frac{\varepsilon^3}{\exp[\varepsilon/(k_BT)]-1} -E_\varepsilon\right).
\end{equation}
where the effective absorption coefficient, $a^\prime_\varepsilon$, is indroduced, which is corrected to account for stimulated emission:
\begin{equation}
a^\prime_\varepsilon = a_\varepsilon\left(1-\exp\left[-\frac{\varepsilon}{k_BT}\right]\right)
\end{equation}
Following \cite{ZR} we can make now two important conclusions. First, the assoption 
coefficient to be used in simulationg the radiation transport should be corrected for 
the stimulated emission, by means of applying a simple correction factor. Second, the 
spontaneous emission from a plasma with the equilibrium distribution over the energy 
states is related to the corrected absorption coefficient in sunch a way that their ratio,
\begin{equation}\label{eq:mg12}
E_\varepsilon^{(Pl)}(T,\varepsilon)=\frac{8\pi}{h^3c^3}\frac{\varepsilon^3}{\exp[\varepsilon/(k_BT)]-1},
\end{equation}
is the spectral energy density distribution of the black body radiation (the Planckian). We will also
apply it in a normalized form as follows:
\begin{equation}\label{eq:mg13}
E_\varepsilon^{(Pl)}(T,\varepsilon)=\alpha T^4 \frac{15}{\pi^4}\frac{x^3}{\exp(x)-1}\frac{1}{k_BT}, \qquad \alpha=\frac{8\pi^5 k_B^4}{15 h^3c^3},\qquad x=\frac{\varepsilon}{k_BT}.
\end{equation} 
The spectral function in (\ref{eq:mg12}) is normalized by a unity:
$$
\int_0^\infty{\frac{15}{\pi^4}\frac{x^3}{\exp(x)-1}\frac{d\varepsilon}{k_BT}}=\frac{15}{\pi^4}\int_0^\infty{\frac{x^3dx}{\exp(x)-1}}=1,
$$
about the intregrals like this see \S58 in \cite{ll}. Therefore the total energy density in the Planck spectrum equals $\alpha T^4$, as it should.

\section{Bound-bound absroption}
TBC
