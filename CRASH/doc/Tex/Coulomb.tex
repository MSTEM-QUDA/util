\documentclass[english,12pt]{revtex4}
\usepackage[T1]{fontenc}
\usepackage[latin9]{inputenc}
\usepackage{nicefrac}

\begin{document}
\title{{\Large Coulomb correction to the thermodinamic parameters of dense plasmas}}
\maketitle

{\bf Coulomb interactions} we account within the Madelung approximation. The extra term in the free energy which accounts for the
electrostatic energy of each of the ions in the charge state $i$ coupled with i electrons, the latter being uniformly
distributed over the "iono-sphere"
(see Eq.(3.50) in \cite{drake}):
\begin{equation}\label{fterm}
F_M=-\frac{9}{10} \frac{q_e^2}{r_{iono}} \sum_{i=0}^{i_{max}} i^2 N_i,\qquad
r_{iono} = \left( \frac{4 \pi}{3} n_a \right)^{-\frac13}.
\end{equation}
Accordingly, in the requirement of the ionization equilibrium, $\partial F/\partial N_i - \partial F/\partial N_{i+1} - \partial F/\partial N_e = 0$ (with respect to the reaction $(i)\leftrightarrow(i+1)+e$),
the term $\partial F_{M}/\partial N_i -\partial F_{M}/\partial N_{i+1}$ will 
give the contribution of $\frac{9}{10} \frac{(2i+1)q_e^2}{r_{iono}}$ into the left hand side:
\begin{equation}
-T \log \left[ \frac{g_i}    {N_i}     e^{\sum_{j=0}^{i-1} I_j/T} \right] - \frac{9}{10} \frac{i^2     q_e^2}{r_{iono}}
+T \log \left[ \frac{g_{i+1}}{N_{i+1}} e^{\sum_{j=0}^i     I_j/T} \right] + \frac{9}{10} \frac{(i+1)^2 q_e^2}{r_{iono}}
-\mu_e = 0.
\end{equation}
The solution of the ionization equilibrium, hence, reads:
\begin{equation}
\frac{N_{i+1}}{g_{i+1}} = \frac{N_i}{g_i} \exp\left(-\frac1T \left(I_i - \frac{9}{10} \frac{(2i+1) q_e^2}{r_{iono}} + \mu_e \right)\right),
\end{equation}
or, applying this recursively and reducing the sum $\sum_{j=0}^{i-1} (2j+1) = i^2$,
\begin{equation}\label{pfM}
\frac{N_i}{g_i}=\frac{N_0}{g_0}(g_e)^i \exp \left( \frac{9}{10} \frac{i^2 q_e^2}{T r_{iono}} -\frac{\sum_{j=0}^{i-1}I_j}T \right) .
\end{equation}
This may be interpreted as the ionization potential lowering which results from the Coulomb interaction:
each of the potentials, $I_i$, is reduced by $\frac{9}{10} \frac{(2i+1)q_e^2}{r_{iono}}$.
Energy of the ion of the charge state $i$ is $E_i^* = \sum_{j=0}^{i-1}I_j - \frac{9}{10} \frac{i^2 q_e^2}{r_{iono}}$.
This effect shifts the ionization equilibrium towards higher ionization degrees, for a given temperature and atomic density.

The common multiplier, $\frac{N_0}{g_0}$, in each of Eqs.(\ref{pfM}) may be also represented as $\frac{N_a}{S}$.
From the normalization condition, $\sum N_i = N_a$, we find that $S$ is a statistical sum:
\begin{equation}
S=\sum_{i=0}^{i_{max}} g_i (g_e)^i \exp\left(-\frac{E_i^*}T\right),
\end{equation}
so that:
\begin{equation}\label{ni}
p_i = \frac{N_i}{N_a} = \frac{1}S g_i (g_e)^i \exp \left( -\frac{E_i^*}T \right).
\end{equation}

The full Helmholtz free energy now includes the contribution of the electrostatic field energy as in Eq.(\ref{fterm}):
\begin{equation}\label{fe1}
F=-T
\sum_{i=0}^{i_{max}}{
N_i\log\left[g_i
  \frac{eV}{N_i}\left(\frac{MT}{2\pi \hbar^2}\right)^{3/2}\exp \left(-\sum_{j=0}^{i-1}\frac{I_j}T \right)\right]}+F_e
  -\frac{9}{10} \sum_{i=0}^{i_{max}} N_i \frac{i^2 q_e^2}{r_{iono}}.
\end{equation}  

With the ion partition functions as in Eq.(\ref{ni}) one can rewrite Eq.(\ref{fe1}) in the following form:
\begin{equation}\label{ffullm}
F = -TN_a\log\left[\frac{eV}{N_a}\left(\frac{MT}{2\pi \hbar^2}\right)^{3/2}\right]-TN_a\log S + \Omega_e, 
\end{equation}
where, again, $\Omega_e = F_e - \mu_e N_a \sum i p_i = F_e - \mu_e N_a \langle i \rangle $.

{\bf Differentials along the curve of the ionization equilibrium}
\begin{equation}
\frac{dg_e}{g_e} (\langle \delta^2 i \rangle + ZR^-(g_e)) + \frac{dT}{T^2} \langle \delta i \delta E_i^* \rangle =
\left(\frac32 \frac{dT}T + \frac{dV}V\right) Z + \langle \delta(i^2) \delta i \rangle \frac{3}{10} \frac{q_e^2}{Tr_{iono}}
\frac{dV}{V}.
\end{equation}



\begin{thebibliography}{99}
\bibitem{drake}
R.P.Drake, High Energy Density Physics
\end{thebibliography}

\end{document}







where
and the equation for the electron statistical weigh now reads:
\begin{equation}
Z=\langle i\rangle_{DH}=g_{e1}{\rm Fe}_{1/2}(g_e).
\end{equation} 
Here the subscript $DH$ is to emphasize that both averages and the statistical sum should be calculated with the partition functions as in
Eq.(\ref{pfDH}), which depend not only on $T,g_e$, but also on $R_D$. Again, it is easy to check that the partial derivatives of  
Eq.(\ref{FDH}) over $g_e$ and over $1/R_D$ both vanish.

The way to account for the electrostatic interaction by means of lowering the ionization potentials is the most general and well-known. 
However, it results in more cumbersome calculations, because the partition functions depend on $R_D$ which in turn depends on the partition 
functions. Ar the same time the accuracy of the Debye-Huekel approach is not so high, so that to account for this effect via the 'exact' 
partition functions is in fact not exact at all. Even more applicable are these consideration to the Madelung approximation as we describe below.

That is why we develop all the formulae in series over the electrostatic interaction parameter, $q_e^2/(TR_D)$, keeping only zero and first 
order terms. Specifically, the statictical sum in this approximation, becomes $\log S_{DH}=\log S +\langle i^2+i\rangle q^2_e/(2R_DT)$, averages of any
function of index $i$, $f_i$ are related as follows: $\langle f_i\rangle_{DH}=\langle f_i\rangle+(\langle f_i(i^2+i)\rangle-\langle f_i\rangle\langle 
i^2+i\rangle)q^2_e/(2R_DT)$, and in the expansion for
$1/R_D^2$ we simply change $\langle i^2+i\rangle_{DH}$ for $\langle i^2+i\rangle$. With this simplifications we obtain:
\begin{equation}\label{FDHApprox}
F = -TN_a\log\left[\frac{eV}{N_a}\left(\frac{MT}{2\pi \hbar^2}\right)^{3/2}\right]-TN_a\log S +\Omega_e -\frac{VT}{12\pi R_D^3}, 
\end{equation}
\begin{equation}\label{ZDHApprox}
Z=\langle i\rangle + \left(\langle i(i^2+i)\rangle-\langle i\rangle\langle i^2+i\rangle\right)\frac{q^2_e}{2R_DT} =g_{e1}{\rm Fe}_{1/2}(g_e),
\end{equation}
\begin{equation}
\frac1{R_D^2}  = \frac{4\pi q_e^2N_a\langle i+i^2\rangle}{VT}.
\end{equation}
By differentiating the free energy over temperature we find the internal energy and by differentiating over the volume we find the pressure:
\begin{equation}\label{EDHApprox}
\frac{\cal E}{ n_a}=\frac{3T}2\left(1+ZR^+(g_e)\right)+\langle E_i\rangle+\left(\langle E_i(i^2+i)\rangle-(\langle E_i\rangle+T)\langle i^2+i\rangle\right)\frac{q^2_e}{2R_DT},
\end{equation}
\begin{equation}\label{PDHApprox}
 \frac{P}{n_a}=T\left[1+ZR^+(g_e)\right]-\langle i^2+i\rangle\frac{q^2_e}{6R_D},
\end{equation}

{\bf In the limit of strong plasma non-ideality,} specifically, if the 'ionosphere radius', $a=(3/4\pi n_a)^{1/3}$, is not small as compared with the Debye radius: 
$a\ge R_D$, the  Debye-Huekel theory is not applicable. While treating this limiting case, however, it should be mentioned, that if the condition $a\gg R_D$, or, the same,
$a\ll Z^2q_e^2/T$ is achieved at the cost of high atomic concentration, then the pressure ionization becomes dominant. The latter effect is significant, if the ionosphere 
radius is less than the radius of the dominant ion as in the Thomas-Fermi model: $a\le R_{TF}$. Under this condition the approach of the present would be not directly 
applicable, because the ion energy levels, $E_i$ in this case are strongly affected by the compression, resulting ultimately in the total disappearance of the bound 
states for electron in the limit of highest pressures.  To neglect this effect we have to assume that the density is limited from above:
\begin{equation}
a\gg R_{TF}(Z,i_{\max}).
\end{equation}  
Under this condition, which is essentially the condition for the thermal ionization dominating the pressure ionization, the plasma strong non-ideality may occur only at low 
temperature. In this case we can use the Madelung estimate for the electrostatic energy, $-9i^2 q_e^2/10a$ - see \cite{PD} for more detail. This estimate seems to be the
asymptotic exppression for high $i\gg1$, particularly, for $i=1$ the positive contribution fron the electron-electron interaction estimated as $\sim +i^2$ 
should in fact turn to zero ($=i(i-1)$?). Therefore, while considering the transition from the Debye-Huekel electrostatic interaction energy per atom,  
$-(i^2+i)q_e^2/2R_D$, to the Madelung one we evaluate 
the latter for simlicity as as $-9(i^2+i)q_e^2/10a$. To achieve this, the only thing we pratically need is to change $R_D\rightarrow \max(R_D,5a/9)$ in 
Eqs.(\ref{FDHApprox},\ref{ZDHApprox},\ref{EDHApprox},\ref{PDHApprox}).


