\section{Madelung approximation of electrostatic energy}
{\bf Coulomb interactions} we account within the Madelung approximation. The extra term in the free energy which accounts for the
electrostatic energy of each of the ions in the charge state $i$ coupled with i electrons, the latter being uniformly
distributed over the "iono-sphere"
(see Eq.(3.50) in \cite{drake}):
\begin{equation}\label{fterm}
F_M=-E_M \sum_{i=0}^{i_{max}} i^2 N_i,\qquad E_M=\frac{9}{10} \frac{q_e^2}{r_{iono}},\qquad
r_{iono} = \left( \frac{4 \pi}{3} n_a \right)^{-\frac13}.
\end{equation}
Here the Madelung energy,
\begin{equation}
E_M=\frac9{10}\frac{q_e^2}{r_{iono}}=\frac{1.8Ry}{(r_{iono}/a)},
\end{equation}
characterizes the electrostatic energy related per an atomic cell. It is conveniently expressed in terms of the Rydberg constant,  $Ry=\frac{q_e^2}{2a}\approx 13.60$ eV, as long as
the iono-sphere radius, $r_{iono}$, is related to the Bohr radius, $a=\hbar^2/m_e q_e^2\approx0.5\cdot10^{-10}$ m.

Accordingly, in the requirement of the ionization equilibrium, $\partial F/\partial N_i - \partial F/\partial N_{i+1} - \partial F/\partial N_e = 0$ (with respect to the reaction $(i)\leftrightarrow(i+1)+e$),
the term $\partial F_{M}/\partial N_i -\partial F_{M}/\partial N_{i+1}$ will 
give the contribution of $(2i+1) E_M$ into the left hand side:
\begin{equation}
-T \log \left[ \frac{g_i}    {N_i}     e^{\sum_{j=0}^{i-1} I_j/T} \right] - i^2 E_M
+T \log \left[ \frac{g_{i+1}}{N_{i+1}} e^{\sum_{j=0}^i     I_j/T} \right] + (i+1)^2 E_M
-\mu_e = 0.
\end{equation}
The solution of the ionization equilibrium, hence, reads:
\begin{equation}
\frac{N_{i+1}}{g_{i+1}} = \frac{N_i}{g_i} \exp\left(-\frac1T \left(I_i - (2i+1) E_M + \mu_e \right)\right),
\end{equation}
or, applying this recursively and reducing the sum $\sum_{j=0}^{i-1} (2j+1) = i^2$,
\begin{equation}\label{pfM}
\frac{N_i}{g_i}=\frac{N_0}{g_0}(g_e)^i \exp \left( i^2 \frac{E_M}{T} -\frac{\sum_{j=0}^{i-1}I_j}T \right) .
\end{equation}
This may be interpreted as the ionization potential lowering which results from the Coulomb interaction:
each of the potentials, $I_i$, is reduced by $(2i+1)E_M$.
Energy of the ion of the charge state $i$ is $E_i^* = \sum_{j=0}^{i-1}I_j - i^2 E_M$.
This effect shifts the ionization equilibrium towards higher ionization degrees, for a given temperature and atomic density.

The common multiplier, $\frac{N_0}{g_0}$, in each of Eqs.(\ref{pfM}) may be also represented as $\frac{N_a}{S}$.
From the normalization condition, $\sum N_i = N_a$, we find that $S$ is a statistical sum:
\begin{equation}
S=\sum_{i=0}^{i_{max}} g_i (g_e)^i \exp\left(-\frac{E_i^*}T\right),
\end{equation}
so that:
\begin{equation}\label{ni}
p_i = \frac{N_i}{N_a} = \frac{1}S g_i (g_e)^i \exp \left( -\frac{E_i^*}T \right).
\end{equation}

The full Helmholtz free energy now includes the contribution of the electrostatic field energy as in Eq.(\ref{fterm}):
\begin{equation}\label{fe1}
F=-T
\sum_{i=0}^{i_{max}}{
N_i\log\left[g_i
  \frac{eV}{N_i}\left(\frac{MT}{2\pi \hbar^2}\right)^{3/2}\exp \left(-\sum_{j=0}^{i-1}\frac{I_j}T \right)\right]}+F_e
  -\frac{9}{10} \sum_{i=0}^{i_{max}} N_i \frac{i^2 q_e^2}{r_{iono}}.
\end{equation}  

With the ion partition functions as in Eq.(\ref{ni}) one can rewrite Eq.(\ref{fe1}) in the following form:
\begin{equation}\label{ffullm}
F = -TN_a\log\left[\frac{eV}{N_a}\left(\frac{MT}{2\pi \hbar^2}\right)^{3/2}\right]-TN_a\log S + \Omega_e, 
\end{equation}
where, again, $\Omega_e = F_e - \mu_e N_a \sum i p_i = F_e - \mu_e N_a \langle i \rangle $.

{\bf Differentials along the curve of the ionization equilibrium} obey the equation as follows:
\begin{equation}\label{diffstruct}
A_{g_e} \frac{dg_e}{g_e} = A_V \frac{dV}{V} + A_T \frac{dT}{T},
\end{equation}
\begin{equation}
A_{g_e} = \langle \delta^2 i \rangle + ZR^-(g_e), \qquad
A_T     = \frac32 Z - \frac{\langle \delta i \delta E^*_i \rangle}{T}, \qquad
A_V     = Z + L \langle \delta(i^2) \delta i \rangle,
\end{equation}
where 
\begin{equation}
L   = \frac{E_M}{3T} = \frac35 \frac{Ry[eV]}{T[eV] r_{iono}[a]}.
\end{equation}


Again, we express the result in terms of covariances,
$\langle \delta a \delta b \rangle = \langle (a - \langle a \rangle) (b - \langle b \rangle) \rangle$,
and mean values which are now being calculated
using the modified partition functions.
Differentiation of mean values which is necessary for derivation of the above equation on
differentials is not a complicated problem with the following formula:
\begin{equation}
d \langle f_i \rangle = \left\langle \delta f_i \delta\left( \frac{dp_i}{p_i} \right) \right\rangle,
\end{equation}
where $f_i$ is a function of the only argument $i$, for example, $iE_i$ or $i^2+i$.

{\bf Plasma thermodynamics and Equation-Of-State.} 
While differentiating Eq.(\ref{ffullm}) with respect to $T$ and $V$, again, we see that the derivatives
by $g_e$ from the second and third terms cancel 
each other: $g_e(\partial \log S/\partial g_e)=\langle i\rangle=Z$,
that is evident from $d \log S = \langle d (\log p_i) \rangle$,
and $-g_eg_{e1}{\rm Fe}^\prime_{3/2}(g_e)=g_{e1}{\rm Fe}_{1/2}=Z$.
That is why for the internal energy density,
${\cal E}$, and for the pressure, $P$, we find the general expressions
as follows:
\begin{equation}\label{generalE}
{\cal E} = -\frac{T^2}V\left(\frac{\partial}{\partial T}\left(\frac F T\right)\right)=
{\cal E}_i+{\cal E}_e,\qquad{\cal E}_i=\frac32Tn_a,\qquad
{\cal E}_e = n_a\left[ \frac32 T Z R^+ + \langle E \rangle \right],
\end{equation}
\begin{equation}\label{generalP}
P = -\frac{\partial F}{\partial V}=P_i+P_e,\quad
P_i = n_aT,\quad
P_e = n_a \left[ T ZR^+ - V \langle \frac{\partial E}{\partial V} \rangle \right],
\end{equation}
where we assume $E=E(V)$.

In the above equations we add the Madelung corrections into the energy of the electron gas, ${\cal E}_e$, and
into the pressure of electrons, $P_e$, because those corrections are controlled by the electron temperature.

The thermodynamic derivatives can also be expressed in a general form for $E^*_i=E^*_i(V)$.
In such a way one can find the specific heat in isochoric process, per the unit of volume:
\begin{equation}\label{generalCv}
C_{Ve}=\frac{\partial {\cal E}_e}{\partial T}=n_a\left[\frac{\langle\delta^2 E^*_i \rangle}{T^2}+\frac{15}4ZR^+
-\frac{\left(\frac32Z-\frac{\langle\delta E^*_i \delta i\rangle}T\right)^2}{\langle\delta^2i\rangle+ZR^-}\right],
\end{equation}
the temperature derivative of pressure:
\begin{equation}\label{generalPT}
\frac {\partial P_e}{\partial T}=
n_a\left[
	\frac52 Z R^+ -
	\left( Z+\frac{V}{T} \langle \delta \frac{\partial E^*_i}{\partial V} \delta i \rangle \right)
		\frac{\frac32Z-\frac{\langle\delta E^*_i \delta i\rangle}T}{\langle\delta^2i\rangle+ZR^-} -
	\frac{V}{T^2} \langle \delta \frac{\partial E^*_i}{\partial V} \delta E^*_i \rangle
\right],
\end{equation}
as well as the isothermal compressibility:
\begin{equation}\label{generalCompr}
V\frac{\partial P_e}{\partial V}=
n_a T \left[ -\frac{\left(Z + \frac{V}{T} \langle \delta i \delta \frac{\partial E^*_i}{\partial V} \rangle \right)^2}
{\langle \delta^2 i \rangle + ZR^-} +
\frac{V^2}{T^2} \langle \delta^2 \frac{\partial E^*_i}{\partial V} \rangle -
\frac{V^2}{T} \langle \frac{\partial^2 E^*_i}{\partial V^2} \rangle
\right].
\end{equation}
Again, for simplicity in the above equations the contributions due to ion translational motions,
\begin{equation}
C_{Vi}=\frac32n_a, \qquad
\frac{\partial P_i}{\partial T}=n_a, \qquad
V\frac{\partial P_i}{\partial V}=-n_aT,
\end{equation}
are omitted.

To apply the Madelung theory we calculate the first and the second partial derivatives
of the energy levels over volume:
\begin{equation}
\frac{V}{T} \frac{\partial E^*_i}{\partial V} = L i^2, \qquad
\frac{V^2}{T} \frac{\partial^2 E^*_i}{\partial V^2} = -\frac43 L i^2,
\end{equation}
and substitute them into the general formulas for the thermodynamic variables
and the thermodynamic derivatives:
\begin{eqnarray}
\label{madE}
{\cal E}_e &=& n_a\left[ \frac32 T Z R^+ + \langle E^*_i \rangle \right], \\
\label{madP}
P_e &=& n_aT ( ZR^+ - L \langle i^2 \rangle ), \\
\label{madCv}
C_{Ve} = \frac{\partial {\cal E}_e}{\partial T} &=& n_a\left[\frac{\langle\delta^2E^*_i\rangle}{T^2}+\frac{15}4ZR^+
-\frac{\left(\frac32Z-\frac{\langle\delta E^*_i\delta i\rangle}T\right)^2}{\langle\delta^2i\rangle+ZR^-}\right], \\
\label{madPT}
\frac {\partial P_e}{\partial T} &=&
n_a\left[
	\frac52 Z R^+ -
	(Z+L \langle \delta(i^2) \delta i \rangle)
		\frac{\frac32Z-\frac{\langle\delta E^*_i\delta i\rangle}T}{\langle\delta^2i\rangle+ZR^-} -
	L \frac{\langle \delta E^*_i \delta(i^2) \rangle}T
\right], \\
\label{madCompr}
V\frac{\partial P_e}{\partial V} &=&
n_a T \left[ -\frac{(Z + L \langle \delta(i^2) \delta i \rangle)^2}{\langle \delta^2 i \rangle + ZR^-} +
L \left( \frac43 \langle i^2 \rangle + L \langle \delta^2(i^2) \rangle \right) \right].
\end{eqnarray}

{\bf Simulation results.}
Following is the table showing the values of $Z$ calculated for Xenon
at various electron temperatures
(given in electron-volts -- the value of $k_{B}T_{e}$, where $k_{B}$
is in eV/K) and heavy particle concentrations (given in number of
particles per $cm^{3}$). "no" columns contain the same values
calculated without Coulomb interation taken into account.

\begin{center}
\begin{tabular}{|c||c|c|c|c|c|c|c|c|c|c|c|c|}
\hline
Na[$1/cm^3$] & \multicolumn{2}{|c|}{$10^{18}$} & \multicolumn{2}{|c|}{$10^{19}$} & \multicolumn{2}{|c|}{$10^{20}$} & \multicolumn{2}{|c|}{$10^{21}$} & \multicolumn{2}{|c|}{$10^{22}$} & \multicolumn{2}{|c|}{$10^{23}$}\tabularnewline
\hline
Te[eV] & no & Mad & no & Mad & no & Mad & no & Mad & no & Mad & no & Mad\tabularnewline
\hline
\hline
   5. &     3.4 &     3.4 &     2.8 &     3.0 &     2.2 &     2.5 &     1.3 &     1.9 &     0.6 &     1.4 &     0.2 &     2.9\tabularnewline
\hline
  10. &     6.3 &     6.4 &     5.2 &     5.4 &     4.1 &     4.6 &     3.0 &     3.7 &     1.7 &     3.2 &     0.7 &     3.9\tabularnewline
\hline
  15. &     7.3 &     7.3 &     6.9 &     6.9 &     5.9 &     6.3 &     4.4 &     5.4 &     2.7 &     4.6 &     1.1 &     4.8\tabularnewline
\hline
  20. &     9.2 &     9.3 &     7.7 &     7.9 &     6.9 &     7.1 &     5.6 &     6.5 &     3.6 &     5.7 &     1.6 &     5.6\tabularnewline
\hline
  25. &    11.8 &    12.0 &     9.4 &     9.7 &     7.6 &     8.0 &     6.5 &     7.1 &     4.5 &     6.5 &     2.1 &     6.2\tabularnewline
\hline
  30. &    13.8 &    14.0 &    11.5 &    11.9 &     8.9 &     9.5 &     7.1 &     7.8 &     5.4 &     7.0 &     2.6 &     6.7\tabularnewline
\hline
  35. &    15.7 &    15.9 &    13.3 &    13.7 &    10.4 &    11.3 &     7.8 &     8.9 &     6.0 &     7.6 &     3.1 &     7.2\tabularnewline
\hline
  40. &    17.2 &    17.4 &    14.9 &    15.3 &    12.1 &    12.9 &     8.7 &    10.2 &     6.6 &     8.3 &     3.6 &     7.8\tabularnewline
\hline
  45. &    18.1 &    18.2 &    16.4 &    16.7 &    13.4 &    14.2 &     9.9 &    11.6 &     7.1 &     9.2 &     4.1 &     8.5\tabularnewline
\hline
  50. &    19.0 &    19.2 &    17.4 &    17.7 &    14.7 &    15.5 &    11.1 &    12.9 &     7.6 &    10.3 &     4.6 &     9.3\tabularnewline
\hline
  55. &    20.5 &    20.7 &    18.1 &    18.4 &    15.9 &    16.7 &    12.3 &    14.0 &     8.2 &    11.4 &     5.0 &    10.1\tabularnewline
\hline
  60. &    22.0 &    22.2 &    19.0 &    19.3 &    16.9 &    17.5 &    13.3 &    15.1 &     8.8 &    12.4 &     5.5 &    11.0\tabularnewline
\hline
  65. &    23.3 &    23.5 &    20.1 &    20.6 &    17.6 &    18.1 &    14.3 &    16.0 &     9.6 &    13.4 &     5.9 &    11.9\tabularnewline
\hline
  70. &    24.4 &    24.6 &    21.4 &    21.9 &    18.2 &    18.8 &    15.2 &    16.8 &    10.4 &    14.3 &     6.2 &    12.7\tabularnewline
\hline
  75. &    25.3 &    25.4 &    22.6 &    23.0 &    19.0 &    19.7 &    16.1 &    17.5 &    11.2 &    15.1 &     6.6 &    13.5\tabularnewline
\hline
  80. &    25.7 &    25.7 &    23.6 &    24.0 &    19.8 &    20.7 &    16.8 &    18.0 &    12.0 &    15.8 &     6.9 &    14.2\tabularnewline
\hline
  85. &    25.9 &    25.9 &    24.5 &    24.8 &    20.8 &    21.8 &    17.4 &    18.6 &    12.7 &    16.5 &     7.3 &    14.9\tabularnewline
\hline
  90. &    25.9 &    25.9 &    25.1 &    25.3 &    21.9 &    22.7 &    17.9 &    19.2 &    13.5 &    17.0 &     7.7 &    15.5\tabularnewline
\hline
  95. &    26.0 &    26.0 &    25.5 &    25.6 &    22.8 &    23.6 &    18.5 &    20.0 &    14.2 &    17.5 &     8.0 &    16.0\tabularnewline
\hline
 100. &    26.1 &    26.1 &    25.7 &    25.8 &    23.6 &    24.3 &    19.1 &    20.8 &    14.8 &    18.0 &     8.4 &    16.5\tabularnewline
\hline
 105. &    26.2 &    26.2 &    25.9 &    25.9 &    24.3 &    24.9 &    19.7 &    21.6 &    15.4 &    18.4 &     8.9 &    16.9\tabularnewline
\hline
 110. &    26.4 &    26.5 &    25.9 &    26.0 &    24.8 &    25.3 &    20.5 &    22.4 &    16.0 &    18.9 &     9.3 &    17.3\tabularnewline
\hline
 115. &    26.9 &    27.0 &    26.0 &    26.0 &    25.2 &    25.5 &    21.3 &    23.1 &    16.5 &    19.5 &     9.8 &    17.7\tabularnewline
\hline
 120. &    27.5 &    27.6 &    26.1 &    26.1 &    25.5 &    25.7 &    22.0 &    23.7 &    17.0 &    20.1 &    10.2 &    18.1\tabularnewline
\hline
 125. &    28.3 &    28.4 &    26.2 &    26.3 &    25.7 &    25.8 &    22.7 &    24.3 &    17.5 &    20.7 &    10.7 &    18.4\tabularnewline
\hline
\end{tabular}

\par\end{center}

\clearpage
